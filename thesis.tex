%%%%%
%%
%% Sample document ``thesis.tex''
%%
%% Version: v0.2
%% Authors: Jean Martina, Rok Strnisa, Matej Urbas
%% Date: 30/07/2008
%%
%% Copyright (c) 2008-2011, Rok Strniša, Jean Martina, Matej Urbas
%% License: Simplified BSD License
%% License file: ./License
%% Original License URL: http://www.freebsd.org/copyright/freebsd-license.html
%%%%%

% Available documentclass options:
%
%   <all `report` document class options, e.g.: `a5paper`>
%   withindex   - enables the index. New index entries can be added through `\index{my entry}`
%   glossary    - enables the glossary.
%   techreport  - typesets the thesis in the technical report format.
%   firstyr     - formats the document as a first-year report.
%   times       - uses the `Times` font.
%   backrefs    - add back references in the Bibliography section
%
% For more info see `README.md`
\documentclass[withindex,glossary]{cam-thesis}
% \usepackage{tufte-latex}
\usepackage{classicthesis}

% Citations using numbers
\usepackage[numbers]{natbib}
% dropcaps:
% https://tex.stackexchange.com/questions/250474/how-to-use-fancy-dropcaps-with-pdflatex
\usepackage{Carrickc,Typocaps,lettrine}
\renewcommand\LettrineFontHook{\Typocapsfamily}

%%% % https://tex.stackexchange.com/a/97128
\usepackage[sc,osf]{mathpazo}   % With old-style figures and real smallcaps.
%%% \linespread{1.025}              % Palatino leads a little more leading
%%% % Euler for math and numbers
%%% \usepackage[euler-digits,small]{eulervm}

\usepackage{lmodern} % For a stylish font


%%%%%%%%%%%%%%%%%%%%%%%%%%%%%%%%%%%%%%%%%%%%%%%%%%%%%%%%%%%%%%%%%%%%%%%%%%%%%%%%
%% Thesis meta-information
%%

%% The title of the thesis:
\title{Mechanized  Finite Domain Decision Procedures}
% spanning two lines}

%% The full name of the author (e.g.: James Smith):
\author{Siddharth Bhat}

%% College affiliation:
\college{Churchill College}

%% College shield [optional]:
% \collegeshield{CollegeShields/Christs}
\collegeshield{CollegeShields/Churchill}
% \collegeshield{CollegeShields/Clare}
% \collegeshield{CollegeShields/ClareHall}
% \collegeshield{CollegeShields/CorpusChristi}
% \collegeshield{CollegeShields/Darwin}
% \collegeshield{CollegeShields/Downing}
% \collegeshield{CollegeShields/Emmanuel}
% \collegeshield{CollegeShields/Fitzwilliam}
% \collegeshield{CollegeShields/Girton}
% \collegeshield{CollegeShields/GonCaius}
% \collegeshield{CollegeShields/Homerton}
% \collegeshield{CollegeShields/HughesHall}
% \collegeshield{CollegeShields/Jesus}
% \collegeshield{CollegeShields/Kings}
% \collegeshield{CollegeShields/LucyCavendish}
% \collegeshield{CollegeShields/Magdalene}
% \collegeshield{CollegeShields/MurrayEdwards}
% \collegeshield{CollegeShields/Newnham}
% \collegeshield{CollegeShields/Pembroke}
% \collegeshield{CollegeShields/Peterhouse}
% \collegeshield{CollegeShields/Queens}
% \collegeshield{CollegeShields/Robinson}
% \collegeshield{CollegeShields/Selwyn}
% \collegeshield{CollegeShields/SidneySussex}
% \collegeshield{CollegeShields/StCatharines}
% \collegeshield{CollegeShields/StEdmunds}
% \collegeshield{CollegeShields/StJohns}
% \collegeshield{CollegeShields/Trinity}
% \collegeshield{CollegeShields/TrinityHall}
% \collegeshield{CollegeShields/Wolfson}
% \collegeshield{CollegeShields/CUniNoText}
% \collegeshield{CollegeShields/FitzwilliamRed}

%% Submission date [optional]:
% \submissiondate{November, 2042}

%% You can redefine the submission notice [optional]:
% \submissionnotice{A badass thesis submitted on time for the Degree of PhD}

%% Declaration date:
\date{May 2027}

%% PDF meta-info:
\subjectline{Computer Science}
\keywords{one two three}



%%%%%%%%%%%%%%%%%%%%%%%%%%%%%%%%%%%%%%%%%%%%%%%%%%%%%%%%%%%%%%%%%%%%%%%%%%%%%%%%
%% Abstract:
%%
\abstract{%
  My abstract ...
}



%%%%%%%%%%%%%%%%%%%%%%%%%%%%%%%%%%%%%%%%%%%%%%%%%%%%%%%%%%%%%%%%%%%%%%%%%%%%%%%%
%% Acknowledgements:
%%
\acknowledgements{%
  My acknowledgements ...
}



%%%%%%%%%%%%%%%%%%%%%%%%%%%%%%%%%%%%%%%%%%%%%%%%%%%%%%%%%%%%%%%%%%%%%%%%%%%%%%%%
%% Glossary [optional]:
%%
\newglossaryentry{HOL}{
    name=HOL,
    description={Higher-order logic}
}



%%%%%%%%%%%%%%%%%%%%%%%%%%%%%%%%%%%%%%%%%%%%%%%%%%%%%%%%%%%%%%%%%%%%%%%%%%%%%%%%
%% Contents:
%%
\begin{document}



%%%%%%%%%%%%%%%%%%%%%%%%%%%%%%%%%%%%%%%%%%%%%%%%%%%%%%%%%%%%%%%%%%%%%%%%%%%%%%%%
%% Title page, abstract, declaration etc.:
%% -    the title page (is automatically omitted in the technical report mode).
\frontmatter{}



%%%%%%%%%%%%%%%%%%%%%%%%%%%%%%%%%%%%%%%%%%%%%%%%%%%%%%%%%%%%%%%%%%%%%%%%%%%%%%%%
%% Thesis body:
%%
\chapter{Introduction}

\lettrine{O}{nce} upon a time
% This document was created with the help of a custom class file~\cite{example}. A
% {\em \LaTeX{} class file}\index{\LaTeX{} class file@LaTeX class file} is a file,
% which holds style information for a particular \LaTeX{} class\footnote{You can
% find more about classes at \url{http://www.ctan.org/pkg/clsguide}.}.
% 
% There are some handy options for citing publications. It is possible to print just the year of some publication:~\citeyear{example}. It is also possible to print the name of the author(s) in the form ``Author et al.'': \citeauthor{example}. Finally, there is also a command to print the full list of authors of a publication: \citet*{example}.
% 
% This is an example glossary reference: \GLS{HOL}. \\
 

\chapter{2-adic algorithms for non-linear bit-vector arithmetic}

% Solving Satisfiability of Polynomial Formulas By Sample-Cell Projection:  https://arxiv.org/pdf/2003.00409
See that 2-adics are decidable by Dubashi et. al. and that we can use this to
decide non-linear bit-vector arithmetic.


\chapter{Better Model Guided Generalization for Exists Forall in QF\_BV}

Hydra algorithm is naive, find better model guided generalization,
using ideas from virtual substitution for the reals.

\chapter{Ackermannization and Multiplication Abstraction}
Steal from Bitwuzla, mechanize.

\chapter{Reducing Bounded NLIA to QF\_BV}

\chapter{Extending MBA to Inequational Fragment plus Disjunction}

- Unsigned Inequations can be handled using the same idea, see hacker's delight, Ch2.2
- Can disjunction be handled the same way? unclear

\chapter{Multiple-Widths in Width Generic Automataion}

Key idea: If we have widths $x : w$ and $ y : w+k$, note that this really means that
if we are at width $w$, then the high bits of $x$ are zero when treated as a transducer.
So this should allow us to exploit the structure to decide properties.

\chapter{Exponentition, Modulus, and Monus}

See that if we have $2^{w-k}$ and $2^{w - l}$, we really can work in a regime where $w \geq k, w \geq l$,
and just bitblast the case where $w \leq max(k, l)$.
WLOG, $k \geq l$, then we can set $k = l + \delta$, and pick $w' \geq l$.



%%%%%%%%%%%%%%%%%%%%%%%%%%%%%%%%%%%%%%%%%%%%%%%%%%%%%%%%%%%%%%%%%%%%%%%%%%%%%%%%
%% References:
%%
% If you include some work not referenced in the main text (e.g. using \nocite{}), consider changing "References" to "Bibliography".
%

% \renewcommand to change default "Bibliography" to "References"
\renewcommand{\bibname}{References}
\cleardoublepage
\phantomsection
\addcontentsline{toc}{chapter}{References}
\bibliographystyle{plainnat}
\bibliography{thesis}



%%%%%%%%%%%%%%%%%%%%%%%%%%%%%%%%%%%%%%%%%%%%%%%%%%%%%%%%%%%%%%%%%%%%%%%%%%%%%%%%
%% Appendix:
%%

\appendix

\chapter{Extra Information}
Some more text ...



%%%%%%%%%%%%%%%%%%%%%%%%%%%%%%%%%%%%%%%%%%%%%%%%%%%%%%%%%%%%%%%%%%%%%%%%%%%%%%%%
%% Index:
%%
\printthesisindex

\end{document}
